% Options for packages loaded elsewhere
\PassOptionsToPackage{unicode}{hyperref}
\PassOptionsToPackage{hyphens}{url}
\PassOptionsToPackage{dvipsnames,svgnames,x11names}{xcolor}
%
\documentclass[
  a4paper]{article}

\usepackage{amsmath,amssymb}
\usepackage{iftex}
\ifPDFTeX
  \usepackage[T1]{fontenc}
  \usepackage[utf8]{inputenc}
  \usepackage{textcomp} % provide euro and other symbols
\else % if luatex or xetex
  \usepackage{unicode-math}
  \defaultfontfeatures{Scale=MatchLowercase}
  \defaultfontfeatures[\rmfamily]{Ligatures=TeX,Scale=1}
\fi
\usepackage{lmodern}
\ifPDFTeX\else  
    % xetex/luatex font selection
\fi
% Use upquote if available, for straight quotes in verbatim environments
\IfFileExists{upquote.sty}{\usepackage{upquote}}{}
\IfFileExists{microtype.sty}{% use microtype if available
  \usepackage[]{microtype}
  \UseMicrotypeSet[protrusion]{basicmath} % disable protrusion for tt fonts
}{}
\makeatletter
\@ifundefined{KOMAClassName}{% if non-KOMA class
  \IfFileExists{parskip.sty}{%
    \usepackage{parskip}
  }{% else
    \setlength{\parindent}{0pt}
    \setlength{\parskip}{6pt plus 2pt minus 1pt}}
}{% if KOMA class
  \KOMAoptions{parskip=half}}
\makeatother
\usepackage{xcolor}
\usepackage[margin=1in]{geometry}
\setlength{\emergencystretch}{3em} % prevent overfull lines
\setcounter{secnumdepth}{5}
% Make \paragraph and \subparagraph free-standing
\ifx\paragraph\undefined\else
  \let\oldparagraph\paragraph
  \renewcommand{\paragraph}[1]{\oldparagraph{#1}\mbox{}}
\fi
\ifx\subparagraph\undefined\else
  \let\oldsubparagraph\subparagraph
  \renewcommand{\subparagraph}[1]{\oldsubparagraph{#1}\mbox{}}
\fi


\providecommand{\tightlist}{%
  \setlength{\itemsep}{0pt}\setlength{\parskip}{0pt}}\usepackage{longtable,booktabs,array}
\usepackage{calc} % for calculating minipage widths
% Correct order of tables after \paragraph or \subparagraph
\usepackage{etoolbox}
\makeatletter
\patchcmd\longtable{\par}{\if@noskipsec\mbox{}\fi\par}{}{}
\makeatother
% Allow footnotes in longtable head/foot
\IfFileExists{footnotehyper.sty}{\usepackage{footnotehyper}}{\usepackage{footnote}}
\makesavenoteenv{longtable}
\usepackage{graphicx}
\makeatletter
\def\maxwidth{\ifdim\Gin@nat@width>\linewidth\linewidth\else\Gin@nat@width\fi}
\def\maxheight{\ifdim\Gin@nat@height>\textheight\textheight\else\Gin@nat@height\fi}
\makeatother
% Scale images if necessary, so that they will not overflow the page
% margins by default, and it is still possible to overwrite the defaults
% using explicit options in \includegraphics[width, height, ...]{}
\setkeys{Gin}{width=\maxwidth,height=\maxheight,keepaspectratio}
% Set default figure placement to htbp
\makeatletter
\def\fps@figure{htbp}
\makeatother

\makeatletter
\makeatother
\makeatletter
\makeatother
\makeatletter
\@ifpackageloaded{caption}{}{\usepackage{caption}}
\AtBeginDocument{%
\ifdefined\contentsname
  \renewcommand*\contentsname{Table of contents}
\else
  \newcommand\contentsname{Table of contents}
\fi
\ifdefined\listfigurename
  \renewcommand*\listfigurename{List of Figures}
\else
  \newcommand\listfigurename{List of Figures}
\fi
\ifdefined\listtablename
  \renewcommand*\listtablename{List of Tables}
\else
  \newcommand\listtablename{List of Tables}
\fi
\ifdefined\figurename
  \renewcommand*\figurename{Figure}
\else
  \newcommand\figurename{Figure}
\fi
\ifdefined\tablename
  \renewcommand*\tablename{Table}
\else
  \newcommand\tablename{Table}
\fi
}
\@ifpackageloaded{float}{}{\usepackage{float}}
\floatstyle{ruled}
\@ifundefined{c@chapter}{\newfloat{codelisting}{h}{lop}}{\newfloat{codelisting}{h}{lop}[chapter]}
\floatname{codelisting}{Listing}
\newcommand*\listoflistings{\listof{codelisting}{List of Listings}}
\makeatother
\makeatletter
\@ifpackageloaded{caption}{}{\usepackage{caption}}
\@ifpackageloaded{subcaption}{}{\usepackage{subcaption}}
\makeatother
\makeatletter
\@ifpackageloaded{tcolorbox}{}{\usepackage[skins,breakable]{tcolorbox}}
\makeatother
\makeatletter
\@ifundefined{shadecolor}{\definecolor{shadecolor}{rgb}{.97, .97, .97}}
\makeatother
\makeatletter
\makeatother
\makeatletter
\makeatother
\ifLuaTeX
  \usepackage{selnolig}  % disable illegal ligatures
\fi
\IfFileExists{bookmark.sty}{\usepackage{bookmark}}{\usepackage{hyperref}}
\IfFileExists{xurl.sty}{\usepackage{xurl}}{} % add URL line breaks if available
\urlstyle{same} % disable monospaced font for URLs
\hypersetup{
  pdftitle={Applied Time Series Assigment},
  colorlinks=true,
  linkcolor={blue},
  filecolor={Maroon},
  citecolor={Blue},
  urlcolor={Blue},
  pdfcreator={LaTeX via pandoc}}

\title{Applied Time Series Assigment}
\author{}
\date{}

\begin{document}
\maketitle
\ifdefined\Shaded\renewenvironment{Shaded}{\begin{tcolorbox}[borderline west={3pt}{0pt}{shadecolor}, enhanced, interior hidden, boxrule=0pt, frame hidden, sharp corners, breakable]}{\end{tcolorbox}}\fi

\renewcommand*\contentsname{Table of contents}
{
\hypersetup{linkcolor=}
\setcounter{tocdepth}{4}
\tableofcontents
}
\hypertarget{theoretical-exercises}{%
\section{Theoretical Exercises}\label{theoretical-exercises}}

\hypertarget{exercise-1-the-ima11-model}{%
\subsection{\texorpdfstring{Exercise 1: \emph{The IMA(1,1)
Model}}{Exercise 1: The IMA(1,1) Model}}\label{exercise-1-the-ima11-model}}

\textbf{1)} Let's find the permanent component and the transitory
component of an IMA(1,1) process:

\[
\Delta y_t = \varepsilon_t + \theta \varepsilon_{t-1}
\]

Expanding:

\begin{align*}
y_t - y_{t-1} & = \varepsilon_t + \theta \varepsilon_{t-1} \\
(1 - L)y_t & = (1 + \theta L) \varepsilon_t
\end{align*}

If we define \(\Theta(L) = 1 + \theta L\), we have, based on the
decomposition proposed by Beveridge and Nelson:

\[
y_t = \Theta(1)(1 - L)^{-1} \varepsilon_t + [\Theta(L) - \Theta(1)](1 - L)^{-1}\varepsilon_t
\]

With \(\Theta(1) = 1 + \theta\), we expand:

\begin{align*}
y_t & = (1 + \theta)(1 - L)^{-1} \varepsilon_t + (1 + \theta L - 1 - \theta)(1 - L)^{-1}\varepsilon_t \\    
& = (1 + \theta)(1 - L)^{-1} \varepsilon_t - \theta \varepsilon_t
\end{align*}

By identification, we obtain that:

\begin{itemize}
    \item $(1 + \theta)(1 - L)^{-1} \varepsilon_t$: \textbf{Permanent Component}    
    \item $\theta \varepsilon_t$: \textbf{Transitory Component}
\end{itemize}

\hypertarget{exercise-2-the-trend-stationary-model}{%
\subsection{\texorpdfstring{Exercise 2: \emph{The trend stationary
model}}{Exercise 2: The trend stationary model}}\label{exercise-2-the-trend-stationary-model}}

Let's consider the trend stationary model
\(Y_t = \alpha + \beta t + \varepsilon_t\) with
\(\varepsilon_t \sim \mathcal{N}(0, \sigma^2)\).

\textbf{1)} Let's compute the variance of \(Y_t\):

\begin{align*}
V(Y_t) & = E(Y_t^2) - E(Y_t)^2 \\
       & = E((\alpha + \beta t + \varepsilon_t)^2) - (\alpha + \beta t)^2 \\
       & = \alpha^2 + \beta^2 t^2 + \sigma^2 + 2 \alpha \beta t + 2 \alpha E(\varepsilon_t) + 2 \beta t E(\varepsilon_t) - \alpha^2 - \beta^2 t^2 - 2 \alpha \beta t \\
       & = \sigma^2.
\end{align*}

Therefore, the variance of \(Y_t\) is time invariant.

Let's show that the effect of \(\varepsilon_t\) on \(Y_t\) dissipates
asymptotically.

By the Bienaymé-Tchebychev inequality, we have:

\begin{align*}
P\left(\left| \varepsilon_t \right|  > (\alpha + \beta t) / \sqrt(t) \right) & \leq \frac{\sigma^2 t}{(\alpha + \beta t)^2}.
\end{align*}

and

\begin{align*}
P\left(\frac{\varepsilon_t}{\alpha + \beta t} > 1/\sqrt{t} \right) & \leq \frac{\sigma^2 t}{(\alpha + \beta t)^2} \\
\end{align*}

Therefore:

\[
P\left(\left|\frac{\varepsilon_t}{\alpha + \beta t}\right| > 0\right) \to 0 \quad \text{as } t \to \infty.
\]

Thus, the effect of \(\varepsilon_t\) on \(Y_t\) dissipates
asymptotically.

\textbf{2)} Let's compute the variance of the forecast error:

\begin{align*}
V(Y_{t+h} - E[Y_{t+h} | It]) & = V(\alpha + \beta(t+h) + \varepsilon_{t+h} - E[\alpha + \beta(t+h) + \varepsilon_{t+h} | It]) \\
                        & = V(\varepsilon_{t+h} - E[\varepsilon_{t+h} | It]) \\
                        & = V(\varepsilon_{t+h}) \\
                        & = \sigma^2.
\end{align*}

So the variance is constant.

\textbf{3)} Let's differentiate the data:

\begin{align*}
Y_{t+1} - Y_t & = \alpha + \beta(t+1) + \varepsilon_{t+1} - (\alpha + \beta t + \varepsilon_t) \\
              & = \beta + \varepsilon_{t+1} - \varepsilon_t.
\end{align*}

The trend has been removed. The characteristic polynomial is
\(1 - z = 0\). So we have \(z = 1 \in \{-1, 1\}\).

Therefore, there is a moving average unit root.

\hypertarget{exercise-3-the-ma2-model}{%
\subsection{\texorpdfstring{Exercise 3: \emph{The MA(2)
Model}}{Exercise 3: The MA(2) Model}}\label{exercise-3-the-ma2-model}}

Consider the process \(Y_t \sim MA(2)\),
\(Y_t = \varepsilon_t - \theta_1 \varepsilon_{t-1} - \theta_2 \varepsilon_{t-2}\),
where \(\varepsilon_t \sim WN(0, \sigma^2)\).

\textbf{1)} Let's compute the mean of \(Y_t\):

\begin{align*}
E[Y_t] & = E[\varepsilon_t - \theta_1 \varepsilon_{t-1} - \theta_2 \varepsilon_{t-2}] \\
       & = 0.
\end{align*}

Let's compute the variance of \(Y_t\):

\begin{align*}
V(Y_t) & = E(Y_t^2) - [E(Y_t)]^2 \\
       & = E(\varepsilon_t^2) + \theta_1^2 E(\varepsilon_{t-1}^2) + \theta_2^2 E(\varepsilon_{t-2}^2) + E[\text{cross terms}] - 0 \\
       & = \sigma^2 + \theta_1^2 \sigma^2 + \theta_2^2 \sigma^2 \\
       & = \sigma^2(1 + \theta_1^2 + \theta_2^2).
\end{align*}

\textbf{2)} Let's compute the autocorrelation function for this process:

\begin{align*}
\gamma_0 & = V(Y_t) = \sigma^2(1 + \theta_1^2 + \theta_2^2),
\\
\gamma_1 & = \text{Cov}(Y_{t+1}, Y_t) \\
         & = \text{Cov}(\varepsilon_{t+1} - \theta_1 \varepsilon_t - \theta_2 \varepsilon_{t-1}, \varepsilon_t - \theta_1 \varepsilon_{t-1} - \theta_2 \varepsilon_{t-2}) \\
         & = -\theta_1 \sigma^2 + \theta_1 \theta_2 \sigma^2,
\\
\gamma_2 & = \text{Cov}(Y_{t+2}, Y_t) \\
         & = \text{Cov}(\varepsilon_{t+2} - \theta_1 \varepsilon_{t+1} - \theta_2 \varepsilon_t, \varepsilon_t - \theta_1 \varepsilon_{t-1} - \theta_2 \varepsilon_{t-2}) \\
         & = -\theta_2 \sigma^2,
\\         
\gamma_k & = 0 \quad \text{for } k \geq 3.
\end{align*}

So we have: \[
\tau_k = \begin{cases}
\frac{\gamma_1}{\sigma^2(1 + \theta_1^2 + \theta_2^2)}, & \text{if } k = 1, \\
\frac{\gamma_2}{\sigma^2(1 + \theta_1^2 + \theta_2^2)}, & \text{if } k = 2, \\
0, & \text{if } k \geq 3.
\end{cases}
\]

\textbf{3)} Let's assume that \(\theta_1 = \frac{5}{6}\) and
\(\theta_2 = \frac{1}{6}\):

\begin{align*}
\gamma_0 & = \frac{38}{36} \sigma^2, \\
\gamma_1 & = \frac{-5}{36} \sigma^2, \\
\gamma_2 & = \frac{-1}{6} \sigma^2.
\end{align*}

So:

\begin{align*}
\tau_1 & = \frac{38 \sigma^2 \times \frac{-6}{\sigma^2} }{36} = \frac{-5}{6}, \\
\tau_2 & = \frac{-5 \sigma^2 \times \frac{-6}{\sigma^2}}{36} = \frac{5}{6}.
\end{align*}

Let's assume that \(\theta_1 = -1\) and \(\theta_2 = 6\):

\begin{align*}
\gamma_0 & = 38 \sigma^2, \\
\gamma_1 & = -5 \sigma^2, \\
\gamma_2 & = -6 \sigma^2.
\end{align*}

We get:

\begin{align*}
\tau_1 & = \frac{-5}{38}, \\
\tau_2 & = \frac{-6}{38}.
\end{align*}

\hypertarget{empirical-exercises}{%
\section{Empirical Exercises}\label{empirical-exercises}}



\end{document}
